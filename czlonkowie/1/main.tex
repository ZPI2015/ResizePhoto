\chapter{Zespołowe przedsięwzięcie}

\begin{itemize}
\item Przygotowanie zdjęć do legitymacji studenckiej. 
\item Projekt jest odpowiedzią na uzasadnioną potrzebę istnienia narzędzia służącego do automatycznej edycji zdjęć, która ułatwi wydruk zdjęć do legitymacji
\end{itemize}

\section{Członkowie zespołu z określeniem funkcji}
\begin{description}
\item[1] Paweł Golonka - kierownik zespołu
\item[2] Karol Liszka - programista C\#	
\item[3] Bartosz Rusinek - tester aplikacji, autor pomocy, itp
\end{description}

\section{Uzasadnienie potrzeby realizacji projektu}
Projekt ma za zadanie pomóc nowym studentom w przygotowaniu zdjęć do legitymacji czy też do innego użytku. Utworzenie stosownej aplikacji pozwoli na ominięcie kosztów związanych z wykonaniem zdjęcia u profesjonalnego fotografa a także umożliwi łatwe przygotowanie zdjęcia do wysłania w formie elektronicznej. Narzędzie to pozwoli również na tworzenie gotowych  bloków zdjęć w wybranej przez użytkownika wielkości.


\section{Cele projektu}

Zespołowe przedsięwzięcie inżynierskie obejmuje przygotowanie aplikacji dla Państwowej Wyższej Szkoły Zawodowej wspomagającej proces przygotowanie fotografii przeznaczonych do legitymacji studenckiej. Program ma pobierać zdjęcie w jednym z wymienionych formatów, a następnie wykadrować wybrany element oraz zapisać go do podanych rozmiarów. 
W dalszym etapie, jego zadaniem będzie utworzenie bloku zdjęć w oparciu o parametry podane przez użytkownika.

\section{Grupy docelowe}
Odbiorcą aplikacji jest PWSZ w Nowym Sączu - Instytut Techniczny a sama aplikacja dedykowana jest dla studentów. 

\section{Zakres projektu}
\textbf{Etapy, które zostaną zrealizowane aby uzyskać postawiony cel}
\begin{enumerate}
\item Wywiad ze zleceniodawcą - zapoznanie się z oczekiwaniami co do aplikacji.
\item Przygotowanie środowiska do pracy z dokumentami \LaTeX oraz C\#.
\item Opracowanie programu w języku C\#
\item Opracowanie plików pomocy i szaty graficznej programu.
\item Kompilacja raportów do formatu PDF, utworzonych w \LaTeX. 
\end{enumerate}


\section{Struktura podziału prac (zadań) - WBS}
\textbf{Hierarchiczna dekompozycja projektu na zadania i aktywności.}
\begin{enumerate}
\item{Wybranie tematu projektu.}
\item Zebranie informacji w wywiadzie ze zleceniodawcą.
\begin{enumerate}
\item Informacje o celach programu.
\item Informacje na temat sposobu działania.
\item Informacje na temat wyniku który ma zostać zwrócony. 
\end{enumerate}
\item{Organizacja pracy}
\begin{enumerate}
\item Wybranie lidera grupy.
\item Wstępny podział obowiązków pośród członków grupy.
\item Określenie języka programowania oraz środowiska w którym program zostanie zaimplementowany.
\end{enumerate}
\item Przygotowanie środowiska
\begin{enumerate}
\item Instalacja i konfiguracja środowiska LaTeX oraz edytora Texmaker.
\item Utworzenie i konfiguracja repozytorium.
\item Instalacja Microsoft Visual Studio 2013.
\end{enumerate}
\item Budowa programu.
\begin{enumerate}
\item Wybór środowiska .NET, środowiska programistycznego oraz utworzenie w nim projektu.
\item Utworzenie interfejsu.
\item Zaimplementowanie wybranych metod.
\item Zdefiniowanie potrzebnych zmiennych.
\item Połączenie metod z elementami graficznymi programu.
\item Kompilacja programu.
\end{enumerate} 
\item Prace finalne.
\begin{enumerate}
\item Testowanie programu.
\item Poprawianie ewentualnych bugów.
\item Testy wtórne.
\end{enumerate}

\end{enumerate}


\section{Diagram sieciowy}
Diagram sieciowy ukazuje zależności czasowe, węzły (aktywności), krawędzie (zależności czasowe).


\section{Harmonogram}
\subsection{Harmonogram prac poszczególnych członków zespołu}
WBS, lub diagram Gantta.


\section{Dokumentacja}
Przygotowanie środowiska do równoległego opracowania dokumentacji projektu i realizacji przydzielonych zadań poszczególnym członkom zespołu projektowego.

\subsection[Edycja plików dokumentacyjnych]{Edycja plików dokumentacyjnych - każdy członek zespoły niezależnie}
Każdy z członków zespołu edytuje swój plik \LaTeX{} (czlonkowie/nrCzlonka/main.tex) i~umieszcza w nim całość analiz i wyników, które pozwoliły mu zrealizować przydzielone zadanie. Wszystkie pliki graficzne, każdy niezależnie umieszcza w swoim katalogu (czlonkowie/nrCzlonka).

Pierwszą linia w pliku (czlonkowie/nrCzlonka/main.tex), zawiera imię i nazwisko opracowującego członka zespołu:
\begin{lstlisting}
\end{lstlisting}

Każde działanie/zadanie należy DOKŁADNIE opisać podając w poleceniu \s!\zadanieprojektowe! cztery obowiązkowe dane:
\begin{itemize}
\item Rodzaj zadania [Przygotowanie przestrzeni do zespołowej pracy]
\item Data rozpoczęcia [2014-11-01]
\item Data zakończenia [2014-11-02]
\item Aktualny status [zaplanowane do realizacji, w trakcie realizacji, zakończone]
\item dokładny opis realizowanego zadania [powinien zawierać opis, rysunki, tabele, kody napisanych programów]
\end{itemize}

Poniżej znajduje się przykładowy listing dla skróconych dwóch zadań:
\begin{lstlisting}


\end{lstlisting}


\subsubsection{Obsługa SVN}
Precyzyjne instrukcje jak obsługiwać repozytorium i wgrywać zmiany prze poszczególnych członków zespołu.

